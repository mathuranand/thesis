\documentclass[12pt,english]{article}
\usepackage[T1]{fontenc}
\usepackage[latin9]{inputenc}
\usepackage[a4paper]{geometry}
\geometry{verbose,tmargin=1.3in,bmargin=1.3in}
\setlength{\parskip}{\medskipamount}
\setlength{\parindent}{0pt}
\usepackage{float}
\usepackage{amsthm}
\usepackage{amsmath}
\usepackage{amssymb}
\usepackage{graphicx}
\usepackage{setspace}

\usepackage{amsthm}
\usepackage{amssymb}
\usepackage{dsfont}
\DeclareMathOperator{\Res}{Res}
\DeclareMathOperator{\Gr}{Gr}
\DeclareMathOperator{\inte}{int}
\DeclareMathOperator{\di}{d}


\usepackage[round]{natbib}
\usepackage[affil-it]{authblk}

\newcommand{\argmin}{\operatornamewithlimits{argmin}}
\newcommand{\supl}{\operatornamewithlimits{sup}}

\usepackage{kpfonts}
\usepackage{esint}

%%%%%%%%%%%%%%%%%%%%%%%%%%%%%% Textclass specific LaTeX commands.
\theoremstyle{definition}
\newtheorem{defn}{\protect\definitionname}[section]
\theoremstyle{plain}
\newtheorem{thm}{\protect\theoremname}[section]
\theoremstyle{plain}
\newtheorem{cor}{\protect\corollaryname}[section]
\theoremstyle{remark}
\newtheorem{rem}{\protect\remarkname}[section]
\theoremstyle{definition}
\newtheorem{example}{\protect\examplename}[section]
\theoremstyle{plain}
\newtheorem{assumption}{\protect\assumptionname}
\theoremstyle{plain}
\newtheorem{prop}{\protect\propositionname}[section]
\theoremstyle{remark}
\newtheorem{note}{\protect\notename}
\theoremstyle{plain}
\newtheorem{lem}{\protect\lemmaname}[section]
\theoremstyle{remark}
\newtheorem{claim}{\protect\claimname}[section]
\theoremstyle{plain}
\newtheorem*{prop*}{\protect\propositionname}
\theoremstyle{plain}
\newtheorem*{lem*}{\protect\lemmaname}
\theoremstyle{plain}
\newtheorem{fact}{\protect\factname}[section]

\usepackage{babel}
\providecommand{\assumptionname}{Assumption}
\providecommand{\claimname}{Claim}
\providecommand{\definitionname}{Definition}
\providecommand{\examplename}{Example}
\providecommand{\factname}{Fact}
\providecommand{\lemmaname}{Lemma}
\providecommand{\notename}{Note}
\providecommand{\propositionname}{Proposition}
\providecommand{\remarkname}{Remark}
\providecommand{\corollaryname}{Corollary}
\providecommand{\theoremname}{Theorem}


\begin{document}
	
	
	\title{Endogenous technical change and energy demand}
	
	
	\author{Akshay Shanker%
		\thanks{Electronic address: \texttt{akshay.shanker@me.com}}}
	\affil{Centre of Applied Macroeconomic Analysis,\\ Australian National University}
	

	
	
	\date{\today}
	
	\maketitle
	
	
	\maketitle
	
\section{Introduction}

In this note, I will introduce a simple model of endogenous growth with endogenous energy demand to explain stylized trends in energy use. The note is preliminary, technical. 

\section{Endogenous growth model}
\subsection{Model environment}

Output is produced by a representative final goods producer 

\begin{equation*}
Y = \left(1-\alpha\right)^{-1}E^{\gamma\alpha}L^{\alpha\left(1-\gamma\right)} \int_{\mathscr{T}}x\left(i\right)^{1-\alpha}q\left(i\right)^{\alpha}\di i
\end{equation*}

$\mathscr{T}\subset \mathbb{R}^{k}$ is technology space with measure $T$ (a technical detail that becomes handy later). I have suppressed time subscripts to keep the notation simple, I will introduce them when necessary. 
	
Energy ($E$) can be extracted at a cost $E^{\xi}$ where $\xi>0$. 


The function $x$ represents machines used by the final goods producer and $q$ is the quality of the machines. Monopolists produce machines, a monopolist with patent for machine $i$ can produce $x\left(i\right)$ at cost $\left(1-\alpha\right)$

$L$ is labour used by the final goods producer. The total population is $P$ and workers can either work in the final goods sector or do $R\&D$. Here is the labour market clearing condition 

\begin{equation*}
L + S = P
\end{equation*}

People can decide in each period whether to work in the final goods sector as workers and be paid a wage $w$ or do some research, become and entrepreneur. If someone decides to become and entrepreneur, they are randomly allocated by Shiva, the god of creative destruction, to a machine $i\in \mathscr{T}$. Schumpeter then flips a coin that lands on heads with $\eta$ probability, and if the coin is a head, Schumpeter sends the entrepreneur a new machine improved by factor $\Gamma$. That is if an innovation is successful

\begin{equation*}
q\left(i\right)_{t+1} = q\left(i\right)_{t}\left(1+\Gamma\right)
\end{equation*}

The successful entrepreneur gets the monopoly rights for just one period (Schumpeter was actually a socialist). After the patent expires, monopoly rights are re-allocated randomly to the population until a new innovation is made. Note all machine types will be demanded in each period, but if an innovation has occurred in a period then only the leading edge machine will have positive demand. 

The setup I just described may seem absurd, but it is a convenient (see\cite{Acemoglu2012}) way to get balanced growth paths that we can study. We can generalise to a more realistic model later and see if the conclusions we derive are essentially the same. The model will admit a representative household, but since we are not worrying about investment or long lived patents, I will omit the discussion for now. 

\subsection{Equilibrium}

Let's start with final goods producers' demand for machines

\begin{equation*}
p_{x}\left(i\right) = E^{\gamma\alpha}L^{\alpha\left(1-\gamma\right)}q\left(i\right)^{\alpha}x\left(i\right)^{-\alpha}
\end{equation*}

The monopolist who owns the patent to $i$ will control quantity to maximise profits 

\begin{equation*}
\max_{x\left(i\right)} E^{\gamma\alpha}L^{\alpha\left(1-\gamma\right)}q\left(i\right)^{\alpha}x\left(i\right)^{1-\alpha} - \left(1-\alpha\right)x\left(i\right)
\end{equation*}




Taking first order conditions we  get capital goods demand 

\begin{equation*}
x\left(i\right) = E^{\gamma}L^{1-\gamma}q\left(i\right)
\end{equation*}

and monopoly profits  

\begin{equation}
\pi\left(i\right)=\alpha E^{\gamma}L^{1-\gamma}q\left(i\right)
\end{equation}

using this equation we can derive 

\begin{equation}\label{eq: Y1}
Y =   E^{\gamma}L^{1-\gamma}\int_{\mathscr{T}}q\left(i\right)\di i \colon = E^{\gamma}L^{1-\gamma}Q
\end{equation}

Next, the final goods producers' energy demands are (take first order conditions and equate)

\begin{equation*}
E = Y^{\frac{1}{\xi}}\left(\frac{\gamma\alpha}{1-\alpha}\right)^{\frac{1}{\xi}}
\end{equation*}

We can plug this back intro \eqref{eq: Y1} to write output in terms of just technology and labour 

\begin{equation}\label{eq: Y2}
	Y = \left(1-\alpha\right)^{\frac{\xi}{\gamma-\xi}}\left(\frac{\gamma \alpha}{1-\alpha}\right)^{\frac{\gamma}{\xi-\gamma}}L^{\frac{\left(1-\gamma\right)\xi}{\xi-\gamma}}Q^{\frac{\xi}{\xi-\gamma}}\	
\end{equation}


and profits and wages for labour 

\begin{equation}
	\pi\left(i\right) = L^{\frac{(\gamma -1) \xi }{\gamma -\xi }} Q^{\frac{\gamma }{\xi -\gamma }}q\left(i\right) \Lambda_{1}
\end{equation}\

\begin{equation}
w = L^{\frac{\gamma  (\xi -1)}{\gamma -\xi }} Q^{\frac{\xi }{\xi - \gamma }}\Lambda_{2}
\end{equation}


where $\Lambda$ are constants. 

Now let's turn our attention to the innovation decision. In each period, someone decides to either work or become an entrepreneur. If an innovation is successful, they are returned with a patent for the improved product in the next period. Then the expected profitability of becoming an entrepreneur is 

\begin{equation}
\mathbb{E}\Pi_{t+1} =  \eta\left(1+\Gamma\right)\int_{\mathscr{T}}L^{\frac{(\gamma -1) \xi }{\gamma -\xi }} Q_{t+1}^{\frac{\gamma }{\xi -\gamma }}q_{t}\left(i\right) \Lambda_{1}\di i  = \eta\left(1+\Gamma\right)L^{\frac{(\gamma -1) \xi }{\gamma -\xi }} Q_{t+1}^{\frac{\gamma }{\xi -\gamma }}Q_{t} \Lambda_{1}
\end{equation}

When the allocation of scientists is $S$, $Q_{t}$ evolves as follows - using law of large number type arguments 

\begin{equation}
Q_{t+1} = \left(1+\eta\Gamma S\right)Q_{t}
\end{equation}

For someone to be indifferent between being an entrepreneur and worker we need $w_{t} = \mathbb{E}\Pi_{t+1}$ and thus 

\begin{equation}
Q^{\frac{\xi}{\xi-\gamma}} = LQ_{t}^{\frac{\gamma}{\xi-\gamma}+1}\eta\left(1+\Gamma\right)\left(1+\eta\Gamma S\right)^{\frac{\gamma}{\xi-\gamma}}\frac{\Lambda_{1}}{\Lambda_{2}}
\end{equation}

this can be simplified to 

\begin{equation}
1 = \left(P-S\right)\eta\left(1+\Gamma\right)\left(1+\eta\Gamma S\right)^{\frac{\gamma}{\xi-\gamma}}\frac{\Lambda_{1}}{\Lambda_{2}}
\end{equation}

This equation characterises our equilibrium allocation of scientists, there is a unique allocation when $P\eta\left(1+\Gamma\right)\frac{\Lambda_{1}}{\Lambda_{2}}>1$ and $\eta\left(1+\Gamma\right)\left(1+\eta\Gamma P\right)^{\frac{\gamma}{\xi-\gamma}}\frac{\Lambda_{1}}{\Lambda_{2}}<1$ or vice versa (mean value theorem). 

\subsubsection{Balanced growth path characteristics}

The growth rate of $Q$ will be $\eta\Gamma S =\colon g_{Q}$. From \eqref{eq: Y2}, the growth rate of output $Y$ will be $\left(1+g_{Q}\right)^{\frac{\xi}{\xi-\gamma}}-1 = g_{y}$. The growth rate of energy will be $\left(1+g_{y}\right)^{\frac{1}{\xi}}-1 = g_{E} $. Clearly energy will grow at a slower rate than output since $\xi>1$ or energy costs are increasing. Moreover, since 

\begin{equation}\label{eq: energyeff}
\frac{E}{Y} = Y^{\frac{1-\xi}{\xi}}\left(\frac{\gamma\alpha}{1-\alpha}\right)^{\frac{1}{\xi}}
\end{equation}

the energy intensity will increase by a rate $\left(1+g_{y}\right)^{\frac{1-\xi}{\xi}}-1$. Energy prices will increase at the rate of growth of output. 


\section{The CES function and energy efficiency}

If we look at \ref{eq: energyeff}, we can see that if $\xi = 1$, i.e. energy prices are constant, energy efficiency remains constant along the BGP. Does this conclusion change if factors and capital are not unit elastic substitutes? Consider the following production function

\begin{equation}\label{eq: product}
Y  = \left( \left(E^{\gamma}Q\right)^{\rho} + x^{\rho}\right)^{\frac{1}{\rho}}
\end{equation}

I have omitted labour to keep the notation simple, but we can imagine a constant term $L^{1-\gamma}$ augmenting $E$. $Q$ is technology and augments $E$ and $L$; but we can imagine $Q$ augments just labour or just energy, it does not matter for our analysis. 

Constructing a general equilibrium environment with endogenous growth paths is somewhat complicated for this model. But we can ask what happens to the ratio $\frac{E}{Y}$ as $Q$ changes using simple comparative statics. 

Suppose capital is supplied at a constant price, if a monopolist supplies capital then the constant price will be a markup over the cost (we also require $\phi>0$). A constant price of capital means that the ratio of capital to output is constant in equilibrium. Taking first order conditions we get 

\begin{equation}\label{eq: caprat}
\frac{X}{Y} = \psi^{\frac{1}{\rho-1}}
\end{equation}

where $\psi$ is the price of capital. Let's call $\left(E^{\gamma}Q\right)$ effective energy services. So output consists of two inputs, effective energy and capital. If $X$ takes up constant share of $Y$, then effective energy services must also make up a constant share of $Y$. We can see this algebraically by subbing in \ref{eq: caprat} into \label{eq: product} and solving out for $Y$. 

The 

\end{document}
